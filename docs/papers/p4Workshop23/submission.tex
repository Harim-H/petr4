% This must be in the first 5 lines to tell arXiv to use pdfLaTeX, which is strongly recommended.
\pdfoutput=1
% In particular, the hyperref package requires pdfLaTeX in order to break URLs across lines.

\documentclass[11pt]{article}

% Remove the "review" option to generate the final version.
\usepackage{random}

% Standard package includes
\usepackage{times}
\usepackage{latexsym}

%\usepackage[utf8]{inputenc}
%\usepackage[T1]{fontenc}
\usepackage{fourier, erewhon}
\usepackage{geometry}
\usepackage{array, caption, floatrow, tabularx, makecell, booktabs}%
\captionsetup{labelfont = sc}
\setcellgapes{3pt}

\usepackage{diagbox}

% For proper rendering and hyphenation of words containing Latin characters (including in bib files)
\usepackage[T1]{fontenc}
% For Vietnamese characters
% \usepackage[T5]{fontenc}
% See https://www.latex-project.org/help/documentation/encguide.pdf for other character sets

% This assumes your files are encoded as UTF8
\usepackage[utf8]{inputenc}
\usepackage{comment}

% This is not strictly necessary, and may be commented out,
% but it will improve the layout of the manuscript,
% and will typically save some space.
\usepackage{microtype}

% If the title and author information does not fit in the area allocated, uncomment the following
%
%\setlength\titlebox{<dim>}
%
% and set <dim> to something 5cm or larger.


% references
\newcommand{\tabref}[1]{\hyperref[tab:#1]{Table~\ref*{tab:#1}}}
\newcommand{\figref}[1]{\hyperref[fig:#1]{Figure~\ref*{fig:#1}}}
\newcommand{\secref}[1]{\hyperref[sec:#1]{Section~\ref*{sec:#1}}}
\newcommand{\defref}[1]{\hyperref[def:#1]{Definition~\ref*{def:#1}}}
\newcommand{\appref}[1]{\hyperref[app:#1]{Appendix~\ref*{app:#1}}}
\newcommand{\chref}[1]{\hyperref[ch:#1]{Chapter~\ref*{ch:#1}}}
\newcommand{\thmref}[1]{\hyperref[thm:#1]{Theorem~\ref*{thm:#1}}}
\newcommand{\lemref}[1]{\hyperref[lem:#1]{Lemma~\ref*{lem:#1}}}
\newcommand{\exref}[1]{\hyperref[ex:#1]{Example~\ref*{ex:#1}}}


\newcommand{\pfour}{P4}
\newcommand{\pc}{p4c}
\newcommand{\petra}{Petr4}
\newcommand{\stf}{stf}
\newcommand{\pversion}{\pfour \textsubscript {16}}


\title{\pfour's Type System Formalization}

% Author information can be set in various styles:
% For several authors from the same institution:
% \author{Author 1 \and ... \and Author n \\
%         Address line \\ ... \\ Address line}
% if the names do not fit well on one line use
%         Author 1 \\ {\bf Author 2} \\ ... \\ {\bf Author n} \\
% For authors from different institutions:
% \author{Author 1 \\ Address line \\  ... \\ Address line
%         \And  ... \And
%         Author n \\ Address line \\ ... \\ Address line}
% To start a seperate ``row'' of authors use \AND, as in
% \author{Author 1 \\ Address line \\  ... \\ Address line
%         \AND
%         Author 2 \\ Address line \\ ... \\ Address line \And
%         Author 3 \\ Address line \\ ... \\ Address line}

%\author{Parisa Ataei \\
%  Cornell University \\\And
%%  \texttt{psa43@cornell.edu} \\\And
%  Ryan Doenges \\
%  Cornell University \\\And
%%  \texttt{rhd89@cornell.edu} \\\And
%  Chris Sommers \\
%  Keysight Technologies \\\And
%%  \texttt{chris.sommers@keysight.com} \\\And
%  Nate Foster \\
%  Cornell University}
%%  \texttt{jnfoster@cs.cornell.edu}}



\begin{document}
\maketitle


%P4 is a dsl. 
%it has spec, ref impl, some form.
%No complete formalization. 
%bad: ambiguity, inconsistency in spec, hard to extend lang.
%working on a formalization of type system. 
%1- surface level IR that adds more type information to surface syntax.
%2- unlike ref impl separate compiler passes and type checking, local type inference
%3- full formalization of p4's type system and show how extending the language is easier

\pfour\ is a domain-specific language for programming packet-processing devices 
which has been successful in making changes in a network more efficiently. 
A language specification\footnote{\url{https://p4.org/p4-spec/docs/P4-16-v-1.2.3.html}}, 
a reference implementation called \pc\footnote{\url{https://github.com/p4lang/p4c}}
%, 
%and a partial formal definition called \petra\footnote{\url{https://github.com/verified-network-toolchain/petr4}} 
accompany \pfour.
%
Unfortunately, it lacks a complete formalization. This shortcoming is highlighted 
every time an issue is filed on P4's GitHub
repository\footnote{\url{https://github.com/p4lang}} 
that discusses either some ambiguity and incompleteness in P4 spec\footnote{Here are some examples of this sort of issues: \url{https://github.com/p4lang/p4-spec/issues/876}, \url{https://github.com/p4lang/p4-spec/issues/1221}, and \url{https://github.com/p4lang/p4-spec/issues/953}.} or some discrepancy 
between P4 spec, \pc, and \petra\footnote{Here is an example: \url{https://github.com/p4lang/p4c/issues/3730}.}. 
%
In fact, solutions have been proposed to find bugs in implementations of P4 that 
use differential testing, translation validation, and model-based testing~\cite{gauntlet,p4diff}.
%
However, even such solutions cannot point at the ambiguity and 
inconsistencies of P4 spec.

Additionally, the lack of complete formalization comes up every a new feature is 
proposed to be added to P4. The process of adding a feature consists of submitting
a proposal to the working language design team. The team discusses the impact of
adding the feature to P4 and if they reach a consensus then the compiler team 
starts the implementation. 
%
However, because the language design team does not have a full formalization of P4
they mainly predicate places where the new feature impacts. Often times during the
implementation unforeseen problems arise that the language design team and the 
compiler team have to address. 

To address this shortcomings, we propose a complete formalization of P4's type systam and build on the work done by Petr4~\cite{petr4}.
Specifically, we will present:

\begin{itemize}
\item A surface-level intermediate representation (IR) that is  
close to P4's syntax but includes more type information within it. 
\item Typing rules that conduct type checking, cast insertion, and local type inference for P4 programs. These rule generate the fully typed P4 programs written in our IR for a given P4 program.
\item Finally, we propose extending P4 with type families and show where this would affect the language using its formalization.
\end{itemize}

%\section*{Acknowledgment}


\bibliography{submission}
\bibliographystyle{acl_natbib}

%\appendix
%
%\section{Example Appendix}
%\label{sec:appendix}
%
%This is an appendix.

\end{document}
