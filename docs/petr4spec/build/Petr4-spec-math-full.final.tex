% MathMode: full, MathRender: png, MathDpi: 300, MathEmbedLimit: 524288, MathScale: 108, MathBaseline: 0, MathDocClass: [10pt]book, MathImgDir: math, MathLatex: pdflatex, Dvipng: dvipng, Convert: convert, Dvips: dvips, Ps2pdf: ps2pdf
\documentclass[10pt]{book}
% generated by Madoko, version 1.2.3
%mdk-data-line={1}
\newcommand\mdmathmode{full}
\newcommand\mdmathrender{png}
\usepackage[heading-base={2},section-num={false},bib-label={true},fontspec={true}]{madoko2}
\usepackage[top=1in, bottom=1.25in, left=1in, right=1in]{geometry}
\usepackage{fancyhdr}
\usepackage{mathpartir}
%mdk-data-line={13}

  \setlength{\headheight}{30pt}
  \renewcommand{\footrulewidth}{0.5pt}

\begin{document}


\begin{mdSnippets}
%mdk-data-line={140}
%mdk-data-line={146}
%mdk-begin-mathdefs
%mdk-data-line={155;ops.tex:1}
%stuff here and there:
\renewcommand{\or}{\ | \ }
\newcommand{\whereBulletIs}{,\textit{ where } \bullet \textit{ is }}
\newcommand{\whereIs}[1]{,\textit{ where } {#1} \textit{ is }}
\newcommand{\where}[1]{,\textit{ where } {#1}}
\newcommand{\textOr}{\textit{ or }}
\newcommand{\numeric}[1]{\mathit{numeric}({#1})}
\newcommand{\error}{\mathit{error}}
\newcommand{\cond}{\mathit{cond}}
\newcommand{\prim}[1]{{#1}^\prime}
\newcommand{\pprim}[1]{{#1}^{\prime\prime}}

%typing judgments:
\newcommand{\envOne}[2]{\Delta, T, \Gamma \vdash {#1} : {#2}}
\newcommand{\expenv}[4]{\env, \ctxt \vdash {#1} \leadsto {#2}, {#3}, {#4}}
\newcommand{\expenvWithCtxt}[5]{\env, {#1} \vdash {#1} \leadsto {#2}, {#3}, {#4}}
\newcommand{\coerceBinArgsEnv}[3]{\env, \ctxt \vdash {#1} \twoheadrightarrow {#2} {#3}}
\newcommand{\binOpEnv}[4]{\env \vdash {#1} \hookrightarrow {#2} {#3} {#4}}

%contexts:
\newcommand{\cte}{\mathit{CONSTANT}}

%metavariables:
\newcommand{\env}{e}
\newcommand{\typ}{t}
\newcommand{\ctxt}{c}
\newcommand{\bool}{b}
\newcommand{\str}{s}
\newcommand{\width}{w}
\newcommand{\val}{v}
\newcommand{\bit}{\mathit{bit}}
\renewcommand{\int}{n}
\newcommand{\bitWidth}[2]{{#1}_{#2}}
\newcommand{\intWidth}[2]{{#1}_{#2}}
\newcommand{\name}{\mathit{name}}
\renewcommand{\array}{a}
\renewcommand{\index}{i}
\newcommand{\arrayAccess}[2]{{#1}[{#2}]}
\newcommand{\size}{n}
\newcommand{\high}{h}
\newcommand{\low}{l}
\newcommand{\bitString}{bs}
\newcommand{\bitStringAccess}[3]{{#1}[{#2}:{#3}]}
\renewcommand{\exp}{\mathit{exp}}
\newcommand{\field}{f}

%function helpers
\newcommand{\isNumeric}[1]{\mathit{is\_numeric}({#1})}
\newcommand{\isArray}[1]{\mathit{is\_array}({#1})}
\newcommand{\compileTimeEval}[1]{[[{#1}]]_\env}

%types:
\newcommand{\bitWidthTyp}[1]{\mathit{bit}<\!{#1}\!>}
\newcommand{\intWidthTyp}[1]{\mathit{int}<\!{#1}\!>}
\newcommand{\boolTyp}{\mathit{bool}}
\newcommand{\stringTyp}{\mathit{string}}
\newcommand{\integerTyp}{\mathit{integer}}
\newcommand{\bitStringTyp}[2]{\mathit{bit[{#1}-{#2}]}}
\newcommand{\intTyp}{\mathit{int}}

%directions:
\newcommand{\less}{\mathit{directionless}}
\newcommand{\dir}{d}
\renewcommand{\in}{\mathit{in}}
\newcommand{\out}{\mathit{out}}



%mdk-data-line={204}
\begin{mdDisplaySnippet}[55d387d31bcd45d0d44edfa68607a2bb]%mdk
%mdk-data-line={209}
\begin{mathpar}
  \small

  \inferrule[Bool]
     {}
     {\expenv \bool \bool \boolTyp \less}

  \inferrule[String]
     {}
     {\expenv \str \str \stringTyp \less}

  \inferrule[Integer]
     {}
     {\expenv \int \int \integerTyp \less}

  \inferrule[Bit]
     {}
     {\expenv {\bitWidth \bit \width} {\bitWidth \bit \width} {\bitWidthTyp \width} \less}

  \inferrule[Int]
     {}
     {\expenv {\intWidth \int \width} {\intWidth \int \width} {\intWidthTyp \width} {\less}}

  \inferrule[Name]
     {\env (\name) = (\typ, \dir)}
     {\expenv \name \name \typ \dir}

  \inferrule[ArrayAccess]
     {\expenv {\arrayAccess \array \index} {\prim \array} {\arrayAccess \typ \size} \dir \\
      \expenv \index {\prim \index} {\prim \typ} {\prim \dir} \\
      \isArray {\typ[\size]} \\
      \isNumeric {\prim \typ}}
     {\expenv {\arrayAccess \array \index} {\arrayAccess {\prim \array} {\prim \index}} \typ \dir }

  \inferrule[BitStringAccess]
     {\expenvWithCtxt \cte \high {\prim \high} {\typ_\high} {\dir_\high} \\
      \isNumeric {\typ_\high} \\
      \pprim \high = \compileTimeEval {\prim \high} \\
      \expenvWithCtxt \cte \low {\prim \low} {\typ_\low} {\dir_\low} \\
      \isNumeric {\typ_\low} \\
      \pprim \low = \compileTimeEval {\prim \low} \\
      0 \leq \pprim \low < \width \\
      \pprim l \leq \pprim h < \width \\
      \expenv \bitString {\prim \bitString} \typ \dir \\
      \typ = \intWidthTyp \width \textOr \bitWidthTyp \width}
     {\expenv {\bitStringAccess \bitString \low \high} {\bitStringAccess \bitString {\pprim \low} {\pprim \high}} {\bitStringTyp {\pprim \low} {\pprim \high}} \dir }

  \inferrule[List]
     {1 \leq i \leq n; \expenv {\exp_i} {\prim {\exp_i}} {\typ_i} {\dir_i}}
     {\expenv {[\exp_1, \ldots, \exp_n]} {[\exp_1, \ldots, \exp_n]} {[\typ_1, \ldots, \typ_n]} \less }

  \inferrule[Record]
     {1 \leq i \leq n; \expenv {\exp_i} {\prim {\exp_i}} {\typ_i} {\dir_i}  }
     {\expenv {\{\field_1 = \exp_1, \ldots, \field_n = \exp_n \}} {\{\field_1 = \prim {\exp_1}, \ldots, \field_n = \prim {\exp_n} \}} {\{\field_1 : \typ_1, \ldots, \field_n : \typ_n \}} \less  }

  \inferrule[LogicalNegation]
     { \expenv \exp {\prim \exp} \boolTyp \dir }
     { \expenv {!\exp} {!\prim \exp} \boolTyp \dir }

  \inferrule[BitwiseComplement]
     { \expenv \exp {\prim \exp} {\bitWidthTyp \width} \dir }
     { \expenv {\sim\!\exp} {\sim\!\prim \exp} {\bitWidthTyp \width} \dir }

  \inferrule[UnaryMinusCTK]
     { \expenv \exp {\prim \exp} \intTyp \dir }
     { \expenv {-\exp} {-\prim \exp} \intTyp \dir }

  \inferrule[UnaryMinus]
     { \expenv \exp {\prim \exp} {\intWidthTyp \width} \dir }
     { \expenv {-\exp} {-\prim \exp} {\intWidthTyp \width} \dir }

  \inferrule[BinaryOps]
     {\coerceBinArgsEnv {\exp_1 \oplus \exp_2} {\prim {\exp_1}} {\prim {\exp_2}} \\
      \binOpEnv {\prim {\exp_1} \oplus \prim {\exp_2}} {\pprim {\exp_1} \oplus \pprim {\exp_2}} \typ \dir }
     { \expenv {\exp_1 \oplus \exp_2} {\prim {\exp_1} \oplus \prim {\exp_2}} \typ \dir }

  % \inferrule[Cast]

\end{mathpar}\end{mdDisplaySnippet}%mdk
%mdk-data-line={291}
\begin{mdDisplaySnippet}[1e2618759b6fabe96b2ac698c983d3c4]%mdk
%mdk-data-line={294}
\begin{mathpar}
  \small

\end{mathpar}\end{mdDisplaySnippet}%mdk
%mdk-data-line={301}
\begin{mdDisplaySnippet}[1e2618759b6fabe96b2ac698c983d3c4]%mdk
%mdk-data-line={304}
\begin{mathpar}
  \small

\end{mathpar}\end{mdDisplaySnippet}%mdk

\end{mdSnippets}

\end{document}
